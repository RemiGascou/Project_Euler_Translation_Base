\section{Multiples of 3 and 5} \label{pb.001}
%PB1
If we list all the natural numbers below 10 that are multiples of 3 or 5, we get 3, 5, 6 and 9. The sum of these multiples is 23.

Find the sum of all the multiples of 3 or 5 below 1000.


\section{Even Fibonacci numbers} \label{pb.002}
%PB2
Each new term in the Fibonacci sequence is generated by adding the previous two terms. By starting with 1 and 2, the first 10 terms will be:

$$1, 2, 3, 5, 8, 13, 21, 34, 55, 89, ...$$

By considering the terms in the Fibonacci sequence whose values do not exceed four million, find the sum of the even-valued terms.


\section{Largest prime factor} \label{pb.003}
%PB3

The prime factors of 13195 are 5, 7, 13 and 29.

What is the largest prime factor of the number 600851475143 ?


\section{Largest palindrome product} \label{pb.004} 
%PB4

A palindromic number reads the same both ways. The largest palindrome made from the product of two 2-digit numbers is $9009 = 91 \times 99$.

Find the largest palindrome made from the product of two 3-digit numbers.


\section{Smallest multiple} \label{pb.005}
%PB5

2520 is the smallest number that can be divided by each of the numbers from 1 to 10 without any remainder.

What is the smallest positive number that is evenly divisible by all of the numbers from 1 to 20?


\section{Sum square difference} \label{pb.006}
%PB6
The sum of the squares of the first ten natural numbers is,
$$1^2 + 2^2 + ... + 10^2 = 385$$

The square of the sum of the first ten natural numbers is,
$$(1 + 2 + ... + 10)^2 = 55^2 = 3025$$

Hence the difference between the sum of the squares of the first ten natural numbers and the square of the sum is $3025 - 385 = 2640$.

Find the difference between the sum of the squares of the first one hundred natural numbers and the square of the sum.


\section{$10 001^{\text{st}}$ prime} \label{pb.007}
%PB7
By listing the first six prime numbers: 2, 3, 5, 7, 11, and 13, we can see that the $6^{\text{th}}$ prime is 13.

What is the $10 001^{\text{ème}}$ prime number?


\section{Largest product in a series} \label{pb.008}
%PB8
The four adjacent digits in the 1000-digit number that have the greatest product are :
$$9 \times 9 \times 8 \times 9 = 5832$$

\begin{center}
    \begin{tabular}{c}
        73167176531330624919225119674426574742355349194934\\
        96983520312774506326239578318016984801869478851843\\
        85861560789112949495459501737958331952853208805511\\
        12540698747158523863050715693290963295227443043557\\
        66896648950445244523161731856403098711121722383113\\
        62229893423380308135336276614282806444486645238749\\
        30358907296290491560440772390713810515859307960866\\
        70172427121883998797908792274921901699720888093776\\
        65727333001053367881220235421809751254540594752243\\
        52584907711670556013604839586446706324415722155397\\
        53697817977846174064955149290862569321978468622482\\
        83972241375657056057490261407972968652414535100474\\
        82166370484403199890008895243450658541227588666881\\
        16427171479924442928230863465674813919123162824586\\
        17866458359124566529476545682848912883142607690042\\
        24219022671055626321111109370544217506941658960408\\
        07198403850962455444362981230987879927244284909188\\
        84580156166097919133875499200524063689912560717606\\
        05886116467109405077541002256983155200055935729725\\
        71636269561882670428252483600823257530420752963450\\
    \end{tabular}
\end{center}

Find the thirteen adjacent digits in the 1000-digit number that have the greatest product. What is the value of this product ?


\section{Special Pythagorean triplet} \label{pb.009}
%PB9
A Pythagorean triplet is a set of three natural numbers, $a < b < c$, for which :
$$a^2 + b^2 = c^2$$

For example, $3^2 + 4^2 = 9 + 16 = 25 = 5^2$.

There exists exactly one Pythagorean triplet for which $a + b + c = 1000$. Find the product $abc$.


\section{Summation of primes} \label{pb.010}
%PB10

The sum of the primes below 10 is 2 + 3 + 5 + 7 = 17.

Find the sum of all the primes below two million.


\newpage


\section{Le plus grand produit dans une grille} \label{pb.011}
%PB11
Dans la grille de $20\times20$ ci-dessous, quatre nombres alignés le long d'une diagonale ont été marqués en rouge.

\begin{center}
    \begin{tabular}{c}
        08 02 22 97 38 15 00 40 00 75 04 05 07 78 52 12 50 77 91 08\\
        49 49 99 40 17 81 18 57 60 87 17 40 98 43 69 48 04 56 62 00\\
        81 49 31 73 55 79 14 29 93 71 40 67 53 88 30 03 49 13 36 65\\
        52 70 95 23 04 60 11 42 69 24 68 56 01 32 56 71 37 02 36 91\\
        22 31 16 71 51 67 63 89 41 92 36 54 22 40 40 28 66 33 13 80\\
        24 47 32 60 99 03 45 02 44 75 33 53 78 36 84 20 35 17 12 50\\
        32 98 81 28 64 23 67 10 \textcolor[rgb]{1,0,0}{26} 38 40 67 59 54 70 66 18 38 64 70\\
        67 26 20 68 02 62 12 20 95 \textcolor[rgb]{1,0,0}{63} 94 39 63 08 40 91 66 49 94 21\\
        24 55 58 05 66 73 99 26 97 17 \textcolor[rgb]{1,0,0}{78} 78 96 83 14 88 34 89 63 72\\
        21 36 23 09 75 00 76 44 20 45 35 \textcolor[rgb]{1,0,0}{14} 00 61 33 97 34 31 33 95\\
        78 17 53 28 22 75 31 67 15 94 03 80 04 62 16 14 09 53 56 92\\
        16 39 05 42 96 35 31 47 55 58 88 24 00 17 54 24 36 29 85 57\\
        86 56 00 48 35 71 89 07 05 44 44 37 44 60 21 58 51 54 17 58\\
        19 80 81 68 05 94 47 69 28 73 92 13 86 52 17 77 04 89 55 40\\
        04 52 08 83 97 35 99 16 07 97 57 32 16 26 26 79 33 27 98 66\\
        88 36 68 87 57 62 20 72 03 46 33 67 46 55 12 32 63 93 53 69\\
        04 42 16 73 38 25 39 11 24 94 72 18 08 46 29 32 40 62 76 36\\
        20 69 36 41 72 30 23 88 34 62 99 69 82 67 59 85 74 04 36 16\\
        20 73 35 29 78 31 90 01 74 31 49 71 48 86 81 16 23 57 05 54\\
        01 70 54 71 83 51 54 69 16 92 33 48 61 43 52 01 89 19 67 48\\
    \end{tabular}
\end{center}

Le produit de ces nombres est $26 \times 63 \times 78 \times 14 = 1788696$.

Quel est le plus grand produit de quatre nombres adjacents dans la même direction (haut, bas, gauche, droite, ou en diagonales) dans une grille de $20\times20$ ?


\section{Diviseurs de nombres triangulaires} \label{pb.012}
%PB12
La séquence des nombres triangulaires est générée par l'addition successive des entiers naturels. Ainsi le $7^{\text{ème}}$ nombre triangulaire est : $1 + 2 + 3 + 4 + 5 + 6 + 7 = 28$. Les dix premiers nombres triangulaires sont donc :

$$1, 3, 6, 10, 15, 21, 28, 36, 45, 55, ...$$

Listons les diviseurs des sept premiers nombres triangulaires :
\begin{center}
    \begin{tabular}{ccccccc}
         1: & 1\\
         3: & 1 & 3\\
         6: & 1 & 2 & 3 & 6\\
        10: & 1 & 2 & 5 & 10\\
        15: & 1 & 3 & 5 & 15\\
        21: & 1 & 3 & 7 & 21\\
        28: & 1 & 2 & 4 & 7 & 14 & 28\\
    \end{tabular}
\end{center}

On remarque que 28 est le premier nombre triangulaire à avoir plus de cinq diviseurs.

Quel est le premier nombre triangulaire à avoir plus de cinq-cent diviseurs ?


\section{Grande somme} \label{pb.013}
%PB13

Quels sont les dix premiers chiffres de la somme des 100 nombres de 50 chiffres suivants ?

$$371...690$$


\section{La plus longue séquence de Collatz} \label{pb.014}
%PB14
La séquence itérative suivante est définie sur l'ensemble des entiers naturels :

\begin{center}
    \begin{tabular}{cc}
        $n \longrightarrow n/2$ & si $n$ est pair.\\
        $n \longrightarrow 3n + 1$ & si $n$ est impair.\\
    \end{tabular}
\end{center}

En utilisant la règle ci-dessus et en commençant avec 13, on obtient la chaîne suivante :
$$13 \rightarrow 40 \rightarrow 20 \rightarrow 10 \rightarrow 5 \rightarrow 16 \rightarrow 8 \rightarrow 4 \rightarrow 2 \rightarrow 1$$

On remarque que cette chaîne (commençant à 13 et finissant à 1) contient 10 termes. Bien que cela n'aie pas pu être démontré à ce jour, (Problème de Collatz), il est supposé que la chaîne arrive à 1 quelque soit le nombre de départ.

Quel nombre de départ, inférieur à un million, produit la plus longue chaîne ?

\textbf{NOTE:} Lorsque la chaîne a été démarrée, les termes peuvent dépasser un million.


\section{Chemins dans un réseau} \label{pb.015}
%PB15 - Lattice paths
%Lattice = Réseau, maillage

En commençant en haut à gauche d'une grille de $2\times2$, et en se déplaçant uniquement vers le bas ou vers la droite, il y a exactement 6 chemins pour aller du coin gauche haut au coin droite bas de la grille.
\begin{center}
    \begin{tikzpicture}
        \draw (-4,5) -- (-2,5)-- (-2,3) -- (-4,3) -- (-4,5);        
        \draw (-3,5) -- (-3,3);
        \draw (-4,4) -- (-2,4);
        \draw[line width=1mm,cap=round,join=round,-angle 45] (-4,5) -- (-2,5) -- (-2,5) -- (-2,3);
        \draw (-1,5) -- (1,5) -- (1,3) -- (-1,3) -- (-1,5);
        \draw (-1,4) -- (1,4);
        \draw (0,5) -- (0,3);
        \draw[line width=1mm,cap=round,join=round,-angle 45] (-1,5) -- (0,5) -- (0,4) -- (1,4) -- (1,3);
        \draw (2,5) -- (4,5) -- (4,3) -- (2,3) -- (2,5);
        \draw (2,4) -- (4,4);
        \draw (3,5) -- (3,3);
        \draw[line width=1mm,cap=round,join=round,-angle 45] (2,5) -- (3,5) -- (3,3) -- (4,3);
        \draw (-4,2) -- (-2,2) -- (-2,0) -- (-4,0) -- (-4,2);
        \draw (-4,1) -- (-2,1);
        \draw (-3,2) -- (-3,0);
        \draw[line width=1mm,cap=round,join=round,-angle 45] (-4,2) -- (-4,1) -- (-2,1) -- (-2,0);
        \draw (-1,2) -- (1,2) -- (1,0) -- (-1,0) -- (-1,2);
        \draw (-1,1) -- (1,1);
        \draw (0,2) -- (0,0);
        \draw[line width=1mm,cap=round,join=round,-angle 45] (-1,2) -- (-1,1) -- (0,1) -- (0,0) -- (1,0);
        \draw (2,2) -- (4,2) -- (4,0) -- (2,0) -- (2,2);
        \draw (2,1) -- (4,1);
        \draw (3,2) -- (3,0);
        \draw[line width=1mm,cap=round,join=round,-angle 45] (2,2) -- (2,0) -- (4,0);
    \end{tikzpicture}
\end{center}

Combien y a-t'il de chemins similaires dans une grille de $20\times20$ ?


\section{Somme des chiffres d'une puissance} \label{pb.016}
%PB16
$2^{15} = 32768$ et la somme de ses chiffres est $3 + 2 + 7 + 6 + 8 = 26$.

Quel est la somme des chiffres du nombre $2^{1000}$?


\section{Nombre de lettres pour écrire un nombre} \label{pb.017}
%PB17

\textbf{\underline{ATTENTION :} Ce problème ne peut être traité qu'en anglais.}
\medskip

Si les nombres de 1 à 5 sont écrits en anglais : \textit{one, two, three, four, five}, alors il y a $3 + 3 + 5 + 4 + 4 = 19$ lettres utilisées au total.

Si tous les nombres de 1 à 1000 (\textit{one thousand}) inclus étaient écrits en anglais, combien de lettres seraient utilisées ?

\textbf{NOTE:} Ne pas compter les espaces ou les traits d'union. Par exemple, 342 (\textit{three hundred and forty-two}) contient 23 lettres et 115 (\textit{one hundred and fifteen}) contient 20 lettres.
L'utilisation du mot \textit{and} dans l'écriture des nombres est en conformité avec l'usage britannique.

\newpage


\section{Somme maximale sur un chemin I} \label{pb.018}
%PB18
En commençant en haut du triangle ci-dessous et en se déplaçant à chaque fois vers les nombres adjacents de la ligne du dessous, la somme maximale du haut vers le bas est de 23.

\begin{center}
    \textcolor[rgb]{1,0,0}{3}\\\textcolor[rgb]{1,0,0}{7} 4\\2 \textcolor[rgb]{1,0,0}{4} 6\\8 5 \textcolor[rgb]{1,0,0}{9} 3\\
\end{center}

On a donc, $3 + 7 + 4 + 9 = 23$.

Quelle est la somme maximale du haut vers le bas du triangle ci-dessous ?

\begin{center}
    75\\95 64\\17 47 82\\18 35 87 10\\20 04 82 47 65\\19 01 23 75 03 34\\88 02 77 73 07 63 67\\99 65 04 28 06 16 70 92\\41 41 26 56 83 40 80 70 33\\41 48 72 33 47 32 37 16 94 29\\53 71 44 65 25 43 91 52 97 51 14\\70 11 33 28 77 73 17 78 39 68 17 57\\91 71 52 38 17 14 91 43 58 50 27 29 48\\63 66 04 68 89 53 67 30 73 16 69 87 40 31\\04 62 98 27 23 09 70 98 73 93 38 53 60 04 23\\
\end{center}

\textbf{NOTE:} Étant donné qu'il n'y a que $2^{14}=16384$ chemins, il est possible de résoudre ce problème en les essayant tous.
Cependant, la même idée est reprise dans le \hyperref[pb.067]{problème 67}, avec un triangle contenant cent lignes. Il ne peut donc pas être résolu par brute force, et requiert une méthode astucieuse ! ;o)


\section{Premiers dimanches durant le vingtième siècle} \label{pb.019}
%PB19
Les informations suivantes vous sont données, mais vous pouvez préférer faire des recherches par vous-même.
\begin{center}
    \begin{tabular}{c}
        \textit{1 Jan 1900 was a Monday.}\\
        \textit{Thirty days has September,}\\
        \textit{April, June and November.}\\
        \textit{All the rest have thirty-one,}\\
        \textit{Saving February alone,}\\
        \textit{Which has twenty-eight, rain or shine.}\\
        \textit{And on leap years, twenty-nine.}\\
    \end{tabular}
\end{center}
    
Une année bissextile se produit chaque année divisible par 4, mais pas sur les siècles sauf si ils sont divisibles par 400. 

Combien de dimanches sont tombés le premier du mois durant le vingtième siècle (Du 1 Jan 1901 au 31 Dec 2000) ?

\newpage


\section{Somme des chiffres d'une factorielle} \label{pb.020}
%PB20
La factorielle d'un entier $n$, (notée $n!$), vaut $n \times (n-1) \times ... \times 3 \times 2 \times 1$

Par exemple, pour $10!$ :
$$10!= 10 \times 9 \times ... \times 3 \times 2 \times 1 = 3628800$$

Et la somme des chiffres du nombre $10!$ est :
$$3 + 6 + 2 + 8 + 8 + 0 + 0 = 27$$

Quelle est la somme des chiffres du nombre $100!$ ?


\section{Nombres amicaux} \label{pb.021}
%PB21
Soit $d(n)$ la somme des diviseurs propres du nombre $n$ (nombres plus petits que $n$ et divisant $n$).
Si $d(a) = b$ et $d(b) = a$, où $a \neq b$, alors $a$ et $b$ forment une paire amicale et chacun des nombres $a$ et $b$ est appelé nombre amical.

Par exemple, les diviseurs propres de $220$ sont $1, 2, 4, 5, 10, 11, 20, 22, 44, 55$ et $110$. Donc :
$$d(220) = 1+2+4+5+10+11+20+22+44+55+110 = 284$$
Les diviseurs propres de $284$ sont $1, 2, 4, 71$ et $142$. Donc $d(284) = 220$.
Donc $220$ et $284$ sont deux nombres amicaux.

Quel est la somme de tous les nombres amicaux situés en dessous de 10000 ?


\section{Scores de prénoms} \label{pb.022}
%PB22
En utilisant \href{https://projecteuler.net/project/resources/p022_names.txt}{names.txt} (Clic droit puis 'Enregistrer sous...'), un fichier texte de 46Ko contenant plus de cinq mille prénoms, en commençant par les trier par ordre alphabétique, on détermine le score du prénom par la multiplication de la valeur alphabétique de chaque prénom par la position de celui-ci dans la liste triée.

Par exemple, en prenant la liste triée par ordre alphabétique, COLIN, qui vaut $3 + 15 + 12 + 9 + 14 = 53$, est le $938^{\text{ème}}$ prénom dans la liste. Ainsi, COLIN obtient un score de $938 \times 53 = 49714$.

Quel est le total de tous les scores des prénoms du fichier ?


\section{Sommes non-abondantes} \label{pb.023}
%PB23
A perfect number is a number for which the sum of its proper divisors is exactly equal to the number. For example, the sum of the proper divisors of $28$ would be $1 + 2 + 4 + 7 + 14 = 28$, which means that $28$ is a perfect number.

A number $n$ is called deficient if the sum of its proper divisors is less than $n$ and it is called abundant if this sum exceeds $n$.

As 12 is the smallest abundant number, $1 + 2 + 3 + 4 + 6 = 16$, the smallest number that can be written as the sum of two abundant numbers is $24$. By mathematical analysis, it can be shown that all integers greater than $28123$ can be written as the sum of two abundant numbers. However, this upper limit cannot be reduced any further by analysis even though it is known that the greatest number that cannot be expressed as the sum of two abundant numbers is less than this limit.

Find the sum of all the positive integers which cannot be written as the sum of two abundant numbers.

\newpage


\section{Permutations lexicographiques} \label{pb.024}
%PB24
Une permutation est un arrangement ordonnée d'objets.
Par exemple, $3124$ est une permutation possible des chiffres $1, 2, 3$ et $4$. Si toutes ces permutations sont triées par ordre alphabétique ou numérique, leur ordre ainsi formé est appelé ordre lexicographique. Les permutations lexicographiques de $0, 1$ et $2$ sont :

\begin{center}
    \begin{tabular}{cccccc}
        012 & 021 & 102 & 120 & 201 & 210\\
    \end{tabular}
\end{center}

Quelle est la millionième permutation lexicographique des chiffres $0, 1, 2, 3, 4, 5, 6, 7, 8$ et $9$ ?


\section{Nombre de Fibonacci à 1000 chiffres} \label{pb.025}
%PB25
La suite de Fibonacci est définie par la relation de récurrence suivante :

\begin{center}
    $F_n = F_{n-1} + F_{n-2}$, où $F_1 = 1$ et $F_2 = 1$.
\end{center}

Par conséquent, les 12 premiers termes de la suite sont :

\begin{center}
    \begin{tabular}{cc}
        $F_1 = 1$ & $F_7 = 13$\\
        $F_2 = 1$ & $F_8 = 21$\\
        $F_3 = 2$ & $F_9 = 34$\\
        $F_4 = 3$ & $F_{10} = 55$\\
        $F_5 = 5$ & $F_{11} = 89$\\
        $F_6 = 8$ & $F_{12} = 144$\\
    \end{tabular}
\end{center}

Le $12^{\text{ème}}$ terme, $F_{12}$, est le premier terme à contenir trois chiffres.

Quel est l'indice du premier terme de la suite de Fibonacci à contenir 1000 chiffres ?
